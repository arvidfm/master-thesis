\section{Introduction}

This template can be found on the conference website. Templates are provided for Microsoft Word\textregistered, and \LaTeX. However, we highly recommend using \LaTeX when preparing your submission. Information for full paper submission is available on the conference website.

\section{Page layout and style}

Authors should observe the following rules for page layout. A highly recommended way to meet these requirements is to use a given template (LibreOffice, Microsoft Word® or L A TEX) and check details against the corresponding example PDF file. Given templates, Microsoft Word\textregistered\ or \LaTeX, can be adapted/imported easily in other software such as LibreOffice, Apple Pages, Lua\LaTeX, and Xe\LaTeX, but please be careful to match the layout of the provided PDF example.

\subsection{Basic layout features}

\begin{itemize}
\item Proceedings will be printed in DIN A4 format. Authors must submit their papers in DIN A4 format.
\item Two columns are used except for the title section and for large figures that may need a full page width.
\item Left and right margin are 20 mm each. 
\item Column width is 80 mm. 
\item Spacing between columns is 10 mm.
\item Top margin is 25 mm (except for the first page which is 30 mm to the title top).
\item Bottom margin is 35 mm.
\item Text height (without headers and footers) is maximum 235 mm.
\item Headers and footers must be left empty.
\item Check indentations and spacings by comparing to this example file (in PDF).
\end{itemize}

\subsubsection{Headings}

Section headings are centered in boldface with the first word capitalized and the rest of the heading in lower case. Sub- headings appear like major headings, except they start at the left margin in the column. Sub-sub-headings appear like sub-headings, except they are in italics and not boldface. See the examples in this file. No more than 3 levels of headings should be used.

\subsection{Text font}

Times or Times Roman font is used for the main text. Font size in the main text must be 9 points, and in the References section 8 points. Other font types may be used if needed for special purposes. It is VERY IMPORTANT that while making the final PDF file, you embed all used fonts!

\LaTeX users: users should use Adobe Type 1 fonts such as Times or Times Roman. These are used automatically by the INTERSPEECH\_v2.sty style file. Authors must not use Type 3 (bitmap) fonts.

\subsection{Figures}

All figures must be centered on the column (or page, if the figure spans both columns). Figure captions should follow each figure and have the format given in Figure~\ref{fig:speech_production}.

Figures should be preferably line drawings. If they contain gray levels or colors, they should be checked to print well on a high-quality non-color laser printer.

Graphics (i.\,e., illustrations, figures) must not use stipple fill patterns because they will not reproduce properly in Adobe PDF. Please use only SOLID FILL COLORS.

Figures which span 2 columns (i.\,e., occupy full page width) must be placed at the top or bottom of the page.

\subsection{Tables}

An example of a table is shown in Table~\ref{tab:example}. The caption text must be above the table.

\begin{table}[th]
  \caption{This is an example of a table}
  \label{tab:example}
  \centering
  \begin{tabular}{ r@{}l  r }
    \toprule
    \multicolumn{2}{c}{\textbf{Ratio}} & 
                                         \multicolumn{1}{c}{\textbf{Decibels}} \\
    \midrule
    $1$                       & $/10$ & $-20$~~~             \\
    $1$                       & $/1$  & $0$~~~               \\
    $2$                       & $/1$  & $\approx 6$~~~       \\
    $3.16$                    & $/1$  & $10$~~~              \\
    $10$                      & $/1$  & $20$~~~              \\
    $100$                     & $/1$  & $40$~~~              \\
    $1000$                    & $/1$  & $60$~~~              \\
    \bottomrule
  \end{tabular}
  
\end{table}

\subsection{Equations}

Equations should be placed on separate lines and numbered. Examples of equations are given below. Particularly,
% 
\begin{equation}
  x(t) = s(f_\omega(t))
  \label{eq1}
\end{equation}
% 
where \(f_\omega(t)\) is a special warping function
% 
\begin{equation}
  f_\omega(t) = \frac{1}{2 \pi j} \oint_C 
  \frac{\nu^{-1k} \mathrm{d} \nu}
  {(1-\beta\nu^{-1})(\nu^{-1}-\beta)}
  \label{eq2}
\end{equation}
% 
A residue theorem states that
% 
\begin{equation}
  \oint_C F(z)\,\mathrm{d}z = 2 \pi j \sum_k \mathrm{Res}[F(z),p_k]
  \label{eq3}
\end{equation}
% 
Applying (\ref{eq3}) to (\ref{eq1}), it is straightforward to see that
% 
\begin{equation}
  1 + 1 = \pi
  \label{eq4}
\end{equation}

Finally we have proven the secret theorem of all speech sciences. No more math is needed to show how useful the result is!

\begin{figure}[t]
  \centering
  here is a picture

  \caption{Schematic diagram of speech production.}
  \label{fig:speech_production}
\end{figure}

\subsection{Information for Word users only}

For ease of formatting, please use the styles listed in Table 2. The styles are defined in this template file and are shown in the order in which they would be used when writing a paper. When the heading styles in Table 2 are used, section numbers are no longer required to be typed in because they will be automatically numbered by Word. Similarly, reference items will be automatically numbered by Word when the ``Reference'' style is used.

\begin{table}[t]
  \caption{Main predefined styles in Word}
  \label{tab:word_styles}
  \centering
  \begin{tabular}{ll}
    \toprule
    \textbf{Style Name}      & \textbf{Entities in a Paper}                \\
    \midrule
    Title                    & Title                                       \\
    Author                   & Author name                                 \\
    Affiliation              & Author affiliation                          \\
    Email                    & Email address                               \\
    AbstractHeading          & Abstract section heading                    \\
    Body Text                & First paragraph in abstract                 \\
    Body Text Next           & Following paragraphs in abstract            \\
    Index                    & Index terms                                 \\
    1. Heading 1             & 1\textsuperscript{st} level section heading \\
    1.1 Heading 2            & 2\textsuperscript{nd} level section heading \\
    1.1.1 Heading 3          & 3\textsuperscript{rd} level section heading \\
    Body Text                & First paragraph in section                  \\
    Body Text Next           & Following paragraphs in section             \\
    Figure Caption           & Figure caption                              \\
    Table Caption            & Table caption                               \\
    Equation                 & Equations                                   \\
    \textbullet\ List Bullet & Bulleted lists                              \\\relax
    [1] Reference            & References                                  \\
    \bottomrule
  \end{tabular}
\end{table}

If your Word document contains equations, you must not save your Word document from ``.docx'' to ``.doc'' because when doing so, Word will convert all equations to images of unacceptably low resolution.

\subsection{Hyperlinks}

For technical reasons, the proceedings editor will strip all active links from the papers during processing. Hyperlinks can be included in your paper, if written in full, e.\,g.\ ``http://www.foo.com/index.html''. The link text must be all black. 
Please make sure that they present no problems in printing to paper.

\subsection{Multimedia files}

The INTERSPEECH organizing committee offers the possibility to submit multimedia files. These files are meant for audio-visual illustrations that cannot be conveyed in text, tables and graphs. Just like you would when including graphics, make sure that you have sufficient author rights to the multimedia materials that you submit for publication. The proceedings media will NOT contain readers or players, so be sure to use widely accepted file formats, such as MPEG, Windows WAVE PCM (.wav) or Windows Media Video (.wmv) using standard codecs.

Your multimedia files must be submitted in a single ZIP file for each separate paper. Within the ZIP file you can use folders and filenames to help organize the multimedia files. In the ZIP file you should include a TEXT or HTML index file which describes the purpose and significance of each multimedia file. From within the manuscript, refer to a multimedia illustration by its filename. Use short file names without blanks for clarity.

The ZIP file you submit will be included as-is in the proceedings media and will be linked to your paper in the navigation interface of the proceedings. Causal Productions (the publisher) and the conference committee will not check or change the contents of your ZIP file.

Users of the proceedings who wish to access your multimedia files will click the link to the ZIP file which will then be opened by the operating system of their computer. Access to the contents of the ZIP file will be governed entirely by the operating system of the user's computer.

\subsection{Page numbering}

Final page numbers will be added later to the document electronically. \emph{Don't make any footers or headers!}


\subsection{References}

The reference format is the standard IEEE one. References should be numbered in order of appearance, for example \cite{Davis80-COP}, \cite{Rabiner89-ATO}, \cite[pp.\ 417--422]{Hastie09-TEO}, and \cite{YourName17-XXX}.

\subsection{Abstract}

The total length of the abstract is limited to 200 words. The abstract included in your paper and the one you enter during web-based submission must be identical. Avoid non-ASCII characters or symbols as they may not display correctly in the abstract book.

\subsection{Author affiliation}

Please list country names as part of the affiliation for each country.

\subsection{Submitted files}

Authors are requested to submit PDF files of their manu­scripts. You can use commercially available tools or for instance http://www.pdfforge.org/products/pdfcreator. The PDF file should comply with the following requirements: (a) there must be no PASSWORD protection on the PDF file at all; (b) all fonts must be embedded; and (c) the file must be text searchable (do CTRL-F and try to find a common word such as ‘the’). The proceedings editors (Causal Productions) will contact authors of non-complying files to obtain a replacement. In order not to endanger the preparation of the proceedings, papers for which a replacement is not provided timely will be withdrawn.

