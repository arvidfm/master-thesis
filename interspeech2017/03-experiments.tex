\section{Experiments}
\label{sec:experiments}

\subsection{Data}
\label{sec:data}

To test our method we use the same data as the 2015 Zero Resource Speech Challenge.
The challenge makes use of two corpora: The Buckeye corpus of conersational English \parencite{buckeyecorpus} and the NCHLT speech corpus of read Xitsonga \parencite{barnard2014nchlt}.
For the challenge only a subset of the data is used, consisting of 12 speakers for a total of 5 hours of data for the Buckeye corpus, and 24 speakers for a total of 2.5 hours of data for the NCHLT Xitsonga corpus.
Additionally provided is voice activity information indicating segments containing clean speech, as well as labels indicating the identity of the speaker.

MFCC features were extracted from the data using a frame window length of \SI{25}{\ms} which was shifted \SI{10}{\ms} for each frame, an FFT resolution of 512 frequency steps, and 40 mel-spaced triangular filter banks.
13 coefficients with both delta and delta-delta features were used.
The MFCCs corresponding to segments with voice activity were clustered using an implementation of a Gaussian mixture model (GMM) provided by scikit-learn \parencite{scikit-learn}.
The GMM was trained using the expectation maximisation algorithm, using $M = 1024$ Gaussians with diagonal covariance matrices, for a maximum of 200 iterations.
After training, posteriorgrams were calculated for each frame.

The unsupervised term discovery yielded 6512 fragments and 3149 clusters for the Buckeye corpus, and 3582 fragments and 1782 clusters for the NCHLT Xitsonga corpus\footnote{The cluster files used in this work were generously provided by Roland Thiollière \parencite{thiolliere2015hybrid} and were generated by Aren Jansen \parencite{jansen2011efficient}.}.
70\% of the same-class and different-class fragment pairs were used for training, with the remaining pairs used for validation to determine when to interrupt the training of the models.

%\subsection{Unsupervised term discovery}
%\label{sec:utd}

% Pairs of similar speech fragments were discovered using the system developed by \textcite{jansen2011efficient}, which serves as a baseline for the second track of the Zero Resource Speech Challenge.
% The system works by calculating the approximate cosine similarity between pairs of frames of two input audio segments, based on discretised random projections of PLP features.
% For efficiency only frames found using an approximate nearest neighbour search are compared, yielding a sparse similarity matrix.
% Stretches of similar frames are then found by searching for diagonals in the similarity matrix, which which are then aligned using dynamic time warping (DTW).
% Pairs of segments with a DTW score above a certain threshold are kept and clustered based on pairwise DTW similarity, resulting in a set of clusters of speech segments, or fragments, thought to be of the same class (e.g.\ word).

% For each cluster every possible pair of fragments was extracted from the collection of posteriorgrams retrieved from the GMM and aligned using DTW, yielding pairs of speech frames belonging to the same class.
% Let $K$ be the total number of pairs of fragments aligned.
% To generate a set of pairs of frames belonging to different classes, $K$ fragments were sampled uniformly from the full collection of fragments.
% For each such fragments, another fragment was sampled uniformly from the fragments belonging to a different cluster.
% When sampling fragments belonging to a different cluster, the sampling was performed using only either fragments spoken by the same speaker, or fragments spoken by a different speaker, with a probability corresponding to the ratio of same-speaker to different-speaker pairs among the same-class fragment pairs.
% The different-class fragment pairs were aligned by simply truncating the longer fragment.

\subsection{Model implementation}
We used $D = 64$ outputs for all models.
The models were trained using AdaMax \parencite{kingma2014adam} with the recommended default parameters. % $\alpha = 0.002$, $\beta_1 = 0.9$ and $\beta_2 = 0.999$.
All frames used for training were shuffled once at the start of training, and a minibatch size of 1000 frame pairs was used.
The models were trained until no improvement had been observed on the held-out validation set for 15 epochs, where one epoch is defined as one complete scan over the training data.

All network models were implemented in Python 3.5 using Theano \parencite{theano} for automatic differentiation and GPU acceleration, librosa \parencite{librosa} for feature extraction, scikit-learn \parencite{scikit-learn} for various utilities, and numba \parencite{numba} for accelerating the dynamic time warping code.
The training was performed on a GeForce GTX TITAN with 6~GB VRAM and 12 Intel i7-5930K cores clocked at 3.50~GHz, with 64~GB RAM.

\subsection{Tuning the hyperparameters}
\label{sec:hypersearch}

\Cref{fig:entropy-penalty} shows the Jensen-Shannon and entropy losses after convergence as a function of $\lambda$, with $\alpha$ fixed at $1$.
We choose $\lambda = 0.1$, as little improvement of the entropy is seen for larger values.
The fact that such a low value suffices to minimise the entropy suggests that the entropy is easy to optimise for.

To find an optimal $\alpha$ we make use of the clusters discovered by the UTD system, choosing the $\alpha$ that maximises the separation of the clusters.
We use the silhouette \parencite{rousseeuw1987silhouettes} as a cluster separation measure, taking the distance between individual fragments to be the DTW score, with the symmetrised KL divergence as the frame-based distance, where each frame is represented as the output of the model being evaluated.
The silhouette for different $\alpha$ with $\lambda$ fixed at $0.1$, calculated on a subset of 1000 clusters, can be seen in \cref{fig:silhouette}.
The optimal value is found to be $\alpha = 1.5$ for both data sets.

\begin{table*}
 \centering
 \caption{\label{tab:abx}ABX and silhouette results for the models described in \cref{sec:model-desc}.}
 \begin{tabular}{llrrrrrr} \toprule
   && \multicolumn{3}{c}{English} & \multicolumn{3}{c}{Xitsonga} \\ \cmidrule(lr){3-5} \cmidrule(lr){6-8}
    & Model & Silhouette & Within & Across & Silhouette & Within & Across \\ \midrule
    Baseline & GMM posteriors & 0.008 & 12.3 & 23.8 & 0.066 & 11.4 & 23.2 \\ \midrule
    Proposed models & Real $\mat W$ ($\alpha = 1$) & 0.093 & 14.1 & 21.2 & 0.119 & 15.8 & 25.1 \\
    & Real $\mat W$ ($\alpha = 1.5$) & 0.108 & 12.8 & 19.8 & 0.146 & 14.0 & 23.2 \\
    & Binary $\mat W$ & 0.124 & 12.0 & 19.3 & 0.170 & 12.7 & 21.9 \\
    & Binary output & 0.010 & 16.5 & 24.6 & 0.014 & 19.4 & 29.2 \\ \midrule
    Related models & Deep JS & -0.058 & 18.1 & 25.6 & 0.150 & 17.5 & 23.5 \\
    & Deep coscos$^2$ (own implementation) & 0.175 & 12.0 & 19.7 & 0.298 & 11.8 & 19.2 \\
    & Deep coscos$^2$ \parencite{thiolliere2015hybrid} & --- & 12.0 & 17.9 & --- & 11.7 & 16.6 \\
    & DPGMM + LDA \parencite{heck2016unsupervised} & --- & 10.6 & 16.0 & --- & 8.0 & 12.6 \\ \bottomrule
 \end{tabular}
\end{table*}


\subsection{Model evaluation}
\label{sec:model-desc}
We train two models, one with $\alpha = 1$ and the other with $\alpha = 1.5$ to measure the influence of reweighting the losses.
Both models are trained with an entropy penalty hyperparameter of $\lambda = 0.1$.
We additionally construct an exact partition using the latter model, by binarising the weight matrix $\mat W$.
Finally, we also evaluate how much performance is retained when further binarising the output of the model with binary $\mat W$.

To compare to the shallow models, and to get an idea of how the JS loss performs in general, we also build a deep network with two hidden layers of $500$ sigmoid units each, with $64$ softmax outputs.
The network is trained using the JS loss with $\alpha = 1$.
As softmax outputs are naturally sparse, we do not enforce any entropy penalty.
For comparison we train the same architecture, albeit with 100 sigmoid outputs instead, using the coscos$^2$ loss of \textcite{synnaeve2014phonetics}.
This is the architecture used by \textcite{thiolliere2015hybrid}.
In place of posteriorgrams we use log-scale outputs of 40 mel-scaled filter banks, normalised to have zero mean and unit variance over the whole data set and with a context of 3 frames on both sides, for a total of 280 values as input to the deep networks.

We evaluate the models on the minimal-pair ABX task \parencite{schatz2013evaluating} using the toolkit provided for the Zero Resource Speech Challenge \parencite{versteegh2015zero}.
For the models with continuous output, the frame-based metric is chosen as the symmetrised Kullback-Leibler divergence (with the model output normalised as necessary).
For the model with binary output, however, we use a distance of 0 for identical and 1 for non-identical vectors.

%%% Local Variables: 
%%% enable-local-variables: t
%%% ispell-local-dictionary: "british"
%%% mode: latex
%%% eval: (flyspell-mode)
%%% eval: (flyspell-buffer)
%%% End: 
