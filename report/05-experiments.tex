% Copyright (C) 2016  Arvid Fahlström Myrman
%
% This program is free software; you can redistribute it and/or modify
% it under the terms of the GNU General Public License as published by
% the Free Software Foundation; either version 2 of the License, or
% (at your option) any later version.
%
% This program is distributed in the hope that it will be useful,
% but WITHOUT ANY WARRANTY; without even the implied warranty of
% MERCHANTABILITY or FITNESS FOR A PARTICULAR PURPOSE.  See the
% GNU General Public License for more details.
%
% You should have received a copy of the GNU General Public License along
% with this program; if not, write to the Free Software Foundation, Inc.,
% 51 Franklin Street, Fifth Floor, Boston, MA 02110-1301 USA.

\chapter{Experiments}
\label{ch:experiments}

This chapter describes the application of the model described in \cref{ch:methods} to the task of unsupervised modelling of speech, and in particular the use of the model to improve posteriorgrams generated from a Gaussian mixture model.
First the experimental setup is described, including the data used, the how the data is processed, and how the models are implemented.
Next, a number of experiments aimed at tuning hyperparameters and comparing models are described.
Finally, the models are evaluated using the ABX discriminability task.

\section{Experimental setup}
\subsection{Data}
The 2015 Zero Resource Speech Challenge makes use of two corpora: The Buckeye corpus of conersational English \parencite{buckeyecorpus} and the NCHLT speech corpus of read Xitsonga \parencite{barnard2014nchlt}.
For the challenge only a subset of the data is used, consisting of 12 speakers for a total of 5 hours of data for the Buckeye corpus, and 24 speakers for a total of 2.5 hours of data for the NCHLT Xitsonga corpus.
Additionally provided is voice activity information indicating segments containing clean speech, as well as labels indicating the identity of the speaker.

\subsection{Generating the posteriorgrams}
\label{sec:posteriorgrams}
MFCCs features were extracted from the data using a frame window length \SI{25}{\ms} which was shifted \SI{10}{ms} for each frame, an FFT resolution of 512 frequency steps, and 40 mel-spaced triangular filter banks.
13 coefficients with both delta and delta-delta features were used.
The MFCCs corresponding to segments with voice activity were clustered using an implementation of a Gaussian mixture model (GMM) provided by scikit-learn \parencite{scikit-learn}.
The GMM was trained using the expectation maximisation algorithm, using $M = 1024$ Gaussians with diagonal covariance matrices, for a maximum of 200 iterations.
After training the posteriorgram for the $n$th frame is constructed as $\mat p_n = (p_n^1, p_n^2, \dots, p_n^{1024})$ where $p_n^i = p(z_i \mid \mat x_n)$ \todo{$z^n_i$ or $z_i$ better?} is the posterior probability of the $i$th class given the $n$th frame.

\subsection{Unsupervised term discovery}
\label{sec:utd}

Pairs of similar speech fragments were discovered using the system developed by \textcite{jansen2011efficient}, which serves as a baseline for the second track of the Zero Resource Speech Challenge.
The system works by calculating the approximate cosine similarity between pairs of frames of two input audio segments, based on discretised random projections of PLP features.
For efficiency only frames found using an approximate nearest neighbour search are compared, yielding a sparse similarity matrix.
Stretches of similar frames are then found by searching for diagonals in the similarity matrix, which which are then aligned using dynamic time warping (DTW).
Pairs of segments with a DTW score above a certain threshold are kept and clustered based on pairwise DTW similarity, resulting in a set of clusters of speech segments, or fragments, thought to be of the same class (e.g.\ word).

This process yielded 6512 fragments and 3149 clusters for the Buckeye corpus, and 3582 fragments and 1782 clusters for the NCHLT Xitsonga corpus\footnote{The cluster files used for this work were generously provided by Roland Thiollière and Aren Jansen.}.
For each cluster every possible pair of fragments was extracted from the collection of posteriorgrams retrieved from the GMM and aligned using DTW, yielding pairs of speech frames belonging to the same class.
Let $K$ be the total number of pairs of fragments aligned.
To generate a set of pairs of frames belonging to different classes, $K$ fragments were sampled uniformly from the full collection of fragments.
For each such fragments, another fragment was sampled uniformly from the fragments belonging to a different cluster.
When sampling fragments belonging to a different cluster, the sampling was performed using only either fragments spoken by the same speaker, or fragments spoken by a different speaker, with a probability corresponding to the ratio of same-speaker to different-speaker pairs among the same-class fragment pairs.
The different-class fragment pairs were aligned by simply truncating the longer fragment.

70\% of the same-class and different-class fragment pairs were used for training, with the remaining pairs used for validation to determine when to interrupt the training of the models.

\subsection{Model implementation}
\todo[inline]{mention server specification? cpu/ram/gpu}

We used $D = 64$ outputs for all models.
The models were trained using AdaMax \parencite{kingma2014adam} with the recommended default parameters $\alpha = 0.002$, $\beta_1 = 0.9$ and $\beta_2 = 0.999$.
All frames used for training were shuffled once at the start of training, and a minibatch size of 1000 frames was used.
The models were trained until no improvement had been observed on a held-out validation set for 15 epochs, where one epoch is defined as one complete scan over the training data.

All network models were implemented in Python 3.5 using Theano \parencite{theano} for automatic differentiation and GPU acceleration, librosa \parencite{librosa} for feature extraction, scikit-learn \parencite{scikit-learn} for various utilities, and numba \parencite{numba} for accelerating various code, in particular dynamic time warping.

\section{Tuning the entropy penalty}
\begin{figure}
  \centering
  \begin{tikzpicture}
    \begin{groupplot}[group style={group size=2 by 1, horizontal sep=2cm},xmin=0,xmax=1,ymin=0,ymax=1,width=5cm,height=4cm]
      \nextgroupplot[title=English,xlabel=Penalty ($\lambda$),ylabel=Divergence/entropy,
      legend style={column sep=10pt},legend entries={JS loss,Same-class loss,Different-class loss,Normalised entropy},
      legend columns=2,legend to name=grouplegend,legend cell align=left]
   \addplot table[x=lambda,y=js-v] {data/entropy_buckeye.txt};
   \addplot table[x=lambda,y=same-js-v] {data/entropy_buckeye.txt};
   \addplot table[x=lambda,y=diff-js-v] {data/entropy_buckeye.txt};
   \addplot table[x=lambda,y=entropy-v] {data/entropy_buckeye.txt};
   
   \nextgroupplot[title=Xitsonga,xlabel=Penalty ($\lambda$),ylabel=Divergence/entropy]
   \addplot table[x=lambda,y=js-v] {data/entropy_xitsonga.txt};
   \addplot table[x=lambda,y=same-js-v] {data/entropy_xitsonga.txt};
   \addplot table[x=lambda,y=diff-js-v] {data/entropy_xitsonga.txt};
   \addplot table[x=lambda,y=entropy-v] {data/entropy_xitsonga.txt};
  \end{groupplot}
  \node[yshift=1.6cm] at ($(group c1r1.north)!.5!(group c2r1.north)$) {\ref{grouplegend}};
\end{tikzpicture}

\caption{\label{fig:entropy-penalty} Effect of varying the entropy penalty for the English (left) and Xitsonga (right) corpora.
The average entropy of the output distribution over the validation samples is shown along with the (root) Jensen-Shannon loss: Both the combined JS loss that is optimised for, and separately for same-class and different-class frame pairs.}
\end{figure}

The entropy penalty $\lambda$ is a free parameter, which is data dependent and must be manually specified.
Ideally, $\lambda$ should be such that the entropy is reduced to a satisfactory degree, without sacrificing the Jensen-Shannon loss.
As both the normalised entropy loss and the Jensen-Shannon loss are bounded between 0 and 1, one might expect the optimal value of $\lambda$ to be in the vicinity of $1$.
We train models using $\lambda \in \{0, 0.05, 0.1, \dots, 0.95\}$ for both the Buckeye and NCHLT Xitsonga corpora.

The final validation errors for each model are reported in \cref{fig:entropy-penalty}.
For both corpora, the entropy drops quickly even for small $\lambda$, suggesting that the entropy is relatively easy to optimise for.
As the entropy penalty is increased, the entropy itself does not decrease; however, the different-class JS loss decreases at the expense of the same-class JS loss. \todo{why?}
For future experiments, a penalty of $\lambda = 0.1$ is used.

\section{Balancing same-class and different-class losses}
When enforcing low entropy in the output distribution, the resulting weight matrix becomes sparse.
For instance, after training the model with $\lambda = 0.1$, and inspecting the row-normalised matrix $\mat W$, we find that the largest element on each row is close to 1: on average across the 1024 rows $0.98$ for English and $0.92$ for Xitsonga.
We can thus inspect $\mat W$ to see how many of the $64$ outputs are actually being used by the model.
We take the sum over each column of $\mat W$.
This sum describes roughly how many inputs are mapped to each output.
We find that for both English and Xitsonga, this sum is above $0.5$ for only a minority of outputs: 11 outputs for English, and 10 outputs for Xitsonga.
For English, where $\mat W$ is particularly sparse, none of the other 53 sums even reach $0.05$.

Thus, it seems that the entropy penalty naturally encourages the model to make use of only a subset of the outputs.
However, the actual number of outputs used is not realistic in terms of how many phonemes one would expect to find in a language; it seems that the same-class loss is forcing too many input classes to merge.
To solve this, we restate the Jensen-Shannon loss function, allowing us to specify how much relative weight to give to the same-class and different-class losses.
Let $B_1 = \{i \in B : c_i = 1\}$ be the subset of same-class frame pairs in the current minibatch, and $B_0 = \{i \in B : c_i = 0\}$ the subset of different-class frame pairs.
We then restate the loss as
\begin{equation}
  \frac{1}{(\alpha + 1)|B_1|} \sum_{i \in B_1} L_{\mathrm{same}}(\mat V; \mat x_i, \mat y_i) + \frac{\alpha}{(\alpha + 1)|B_0|} \sum_{i \in B_0} L_{\mathrm{diff}}(\mat V; \mat x_i, \mat y_i),
\end{equation}
where $L_{\mathrm{same}}$ and $L_{\mathrm{diff}}$ are the same-class and different-class losses defined in \cref{eq:js-loss}.
$\alpha$ is a hyperparameter specifying how much more to weight the different-class loss over the same-class loss.

$\alpha$ needs to be carefully tuned: A too small $\alpha$ will cause too many input classes to merge, including classes that correspond to completely different phonemes, while a too large $\alpha$ will cause input classes that do correspond to the same phoneme to fail to merge.
In order to find a good value for $\alpha$, without making use of the gold transcription or prior knowledge of the number of phonemes present in the languages in question, we make use of the fragment clusters discovered by the unsupervised term discovery system.
The intuition is that the goal of our model is to push apart different clusters, while keeping fragments within a cluster as similar as possible.
To measure the success of our model, then, we can make use of a cluster separation measure.

Here we use the silhouette \parencite{rousseeuw1987silhouettes}, which makes use of the average similarity between a sample and every other sample in the same cluster, and between a sample and every sample in the most similar other cluster.
The silhouette ranges from -1 to 1, with a value close to 1 indicating that the clusters are well separated.
Models were trained for $\alpha \in \{1, 1.5, 2, 2.5, 3, 3.5, 4\}$, with an entropy penalty of $\lambda = 0.1$.
The silhouette was then calculated on a subset of 1000 of the fragment clusters, using the output of the trained models to represent the frames of the fragments.
The similarity between to fragments was calculated as the DTW score using the symmetrise Kullback-Leibler divergence as a similarity measure between individual frames.

\todo[inline]{is the below paragraph uninteresting? a bit too model-specific?}
To easily get an estimate of the number of outputs used by the model, we also define the ``spread'' of the model as follows.
We take the average of the $j$th column:
\begin{equation}
  q_j = \frac{1}{M} \sum_{i=1}^M w_{ij}.
\end{equation}
This represents the average mapping to the $i$th output.
As $\mat W$ is row-normalised, the elements of $Q = (q_1, q_2, \dots, q_D)$ sum to 1, and we can thus treat $Q$ as describing a probability distribution.
A uniform distribution means that each output has the same number of inputs mapped to it.
Now consider the case where there are $K$ outputs such that the same number of inputs maps to each output, while no inputs map to any other outputs.
The normalised entropy of $Q$ is then given by
\begin{equation}
  \hat H(Q) = - \frac{1}{\log_2 D} \sum_{i=1}^K \frac{1}{K} \log_2 \frac{1}{K} = \frac{\log_2 K}{\log_2 D}.
\end{equation}
Solving for $K$ we have
\begin{equation}
  K = D^{\hat H(Q)},
\end{equation}
which is an approximation of the number of outputs used by the model, which we define as the spread.
A value of $K$ close to $D$ is an indicator that all the outputs are being used equally, suggesting that it may be a good idea to increase the number of outputs.

\begin{figure}
 \centering
 \begin{tikzpicture}
   \pgfplotsset{set layers}
   \begin{axis}[
     scale only axis,
     xmin=0.8,xmax=4.2,
     ymax=0.25,
     axis y line*=left,
     xlabel=$\alpha$,
     ylabel=Silhouette,
     yticklabel style={/pgf/number format/fixed},
     height=5cm,width=8cm,
     legend style={opacity=0.0}]%,
      %legend style={column sep=10pt},legend entries={Silhouette (English),Silhouette (Xitsonga)},legend cell align=left]
      \addplot table[x=alpha,y=sil-en] {data/silhouette.txt}; \label{sil1}
      \addlegendentry{Silhouette (English)}
   %\addplot+[mark=x] table[x=alpha,y=spread-en] {data/silhouette.txt};
   \addplot table[x=alpha,y=sil-ts] {data/silhouette.txt}; \label{sil2}
   \addlegendentry{Silhouette (Xitsonga)}
   %\addplot+[mark=x] table[x=alpha,y=spread-ts] {data/silhouette.txt};
   \end{axis}
   \begin{axis}[
     scale only axis,
     xmin=0.8,xmax=4.2,
     ymax=80,
     axis y line*=right,
     axis x line=none,
     ylabel=Spread,
     height=5cm,width=8cm,
     legend style={anchor=north west,at={(0.02,0.98)}}]%,
      %legend style={column sep=10pt},legend entries={Silhouette (English),Silhouette (Xitsonga)},legend cell align=left]
   %\addplot+[mark=x] table[x=alpha,y=sil-en] {data/silhouette.txt};
   \addlegendimage{/pgfplots/refstyle=sil1}\addlegendentry{Silhouette (English)}
   \addlegendimage{/pgfplots/refstyle=sil2}\addlegendentry{Silhouette (Xitsonga)}
   \addplot+[dashed] table[x=alpha,y=spread-en] {data/silhouette.txt};
   \addlegendentry{Spread (English)}
   %\addplot+[mark=x] table[x=alpha,y=sil-ts] {data/silhouette.txt};
   \addplot+[dashed] table[x=alpha,y=spread-ts] {data/silhouette.txt};
   \addlegendentry{Spread (Xitsonga)}
   \end{axis}
 \end{tikzpicture}

 \caption{\label{fig:silhouette} Silhouette and spread for different weightings of the same-class and different-class losses.}
\end{figure}

\Cref{fig:silhouette} shows the silhouette and spread for different values of $\alpha$.
As one might expect, more emphasis on the different-class loss results in a higher spread, i.e.\ a larger number of output classes.
The optimal value of $\alpha$ seems to be around 1.5 for both data sets, we use this value of $\alpha$ going forward.

\section{Discretising the model}
As the resulting model is sparse, we can retrieve an exact surjection by discretising the model.
We do this by for each row in $\mat W$ setting the largest element to $1$ and the remaining elements to $0$.
Using the discretised model as a base, we additionally experiment with discretising the output distribution by setting the largest output to 1 and the rest to 0; this can be thought of as taking the argmax of the output distribution.

\section{Comparison with deep models}
To get an idea of how the JS loss performs in general, we build a deep network with two hidden layers of $500$ sigmoid units each, with $64$ softmax outputs.
The network is trained using the non-rebalanced JS loss.
As softmax outputs are naturally sparse, we do not enforce any entropy penalty.
For comparison we train the same architecture, albeit with sigmoid outputs instead, using the coscos$^2$ loss of \textcite{synnaeve2014phonetics}.
This is the architecture used by \textcite{thiolliere2015hybrid} in the 2015 Zero Resource Speech Challenge.

As input to both networks we use the log-scale outputs of 40 mel-scaled filter banks.
All other relevant parameters are the same as for the MFCCs calculated in \cref{sec:posteriorgrams}.
The filter bank outputs are normalised over the whole data set to have zero mean and unit variance for all dimensions.
Each frame is fed to the network with a context of 3 frames on both sides, for a total of 280 values used as input to the network.
All fragments are DTW aligned and sampled as in \cref{sec:utd}.

\section{Interpreting the model}

\section{ABX evaluation}
