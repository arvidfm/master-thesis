% Copyright (C) 2016  Arvid Fahlström Myrman
%
% This program is free software; you can redistribute it and/or modify
% it under the terms of the GNU General Public License as published by
% the Free Software Foundation; either version 2 of the License, or
% (at your option) any later version.
%
% This program is distributed in the hope that it will be useful,
% but WITHOUT ANY WARRANTY; without even the implied warranty of
% MERCHANTABILITY or FITNESS FOR A PARTICULAR PURPOSE.  See the
% GNU General Public License for more details.
%
% You should have received a copy of the GNU General Public License along
% with this program; if not, write to the Free Software Foundation, Inc.,
% 51 Franklin Street, Fifth Floor, Boston, MA 02110-1301 USA.

\chapter{Related work}
\label{ch:related-work}

This \namecref{ch:related-work} provides a brief overview of recent research into unsupervised acoustic modelling.
The approaches discussed here can largely be divided into two categories: bottom-up approaches that infer the acoustic model directly from the speech frames, and top-down approaches that first segment the speech into syllable- or word-like units, and afterwards try break these units into smaller subword units.

\section{Bottom-up approaches}

As an individual speech frame only make up a fraction of a complete speech sound, it is natural to model and segment the speech using a model that can capture time dependencies, such as a hidden Markov model (HMM), rather than attempt to cluster the speech frames directly.
One issue with this approach, however, is that the number of possible states (i.e.\ phonemes) is unknown a priori.

\textcite{varadarajan2008unsupervised} tackle this problem by first defining a one-state HMM, and then iteratively splitting and merging states as needed to account for the data according to a heuristic.
Training stops once the size of the HMM reaches a threshold.
After training, each state in the HMM can be thought to correspond to some allophone (context-dependent variant realisation) of a phoneme.
It should be noted, however, that in order to interpret a given state sequence as a single phoneme, \citeauthor{varadarajan2008unsupervised} train a separate model using labelled speech to perform this mapping.
The method is thus not fully unsupervised.

\textcite{lee2012nonparametric} take a fully probabilistic approach, defining a model that jointly performs segmentation and acoustic modelling.
An infinite mixture model of tri-state HMMs modelling subword units is defined using the Dirichlet process, and latent variables representing segment boundaries are introduced.
The data can be thought to be generated by repeatedly sampling an HMM to model a segment, sampling a path through the HMM, and for each state in the path sampling a feature vector from the corresponding GMM.
The probability of transitioning from one phone to another is thus not modelled.
Inference of the model is done using Gibbs sampling.

\textcite{siu2014unsupervised} use an HMM of a more classic form to model the data.
An initial transcription of the data in terms of state labels is first generated in an unsupervised manner using a segmental GMM (SGMM).
The HMM and transcription are then iteratively updated, maximising the probability of the model parameters given the transcription, and the transcription given the model parameters.
Note that the number of allowed states are here defined in advance.
$n$-gram statistics are then collected from the transcription and used for tasks such as unsupervised keyword discovery.

Diverging from previous approaches using temporal models, \textcite{chen2015parallel} perform standard clustering of speech frames using an infinite Gaussian mixture model.
After training, the speech frames are represented as posteriorgrams, which have been shown to be more speaker-invariant than other features such as MFCCs \parencite{zhang2010towards}.
Despite the simple approach, this turned out to be the overall best-performing model in the first track of the 2015 Zero Resource Speech Challenge \parencite{versteegh2016zero}.
\textcite{heck2016unsupervised} later further improved on the model by performing clustering in two stages, with an intermediate supervised dimensionality reduction step using the clusters derived from the first clustering step as target classes.

\textcite{synnaeve2016temporal} use a siamese network to create an embedding where speech frames close to each other are considered to belong to the same phoneme, while distant speech frames are said to differ.
A siamese network is a feedforward neural network that takes two inputs and adjusts its parameters to either maximise or minimise the similarity of the corresponding outputs \parencite{bromley1994signature}.

\section{Top-down approaches}

Top-down approaches start by first finding pairs of longer word-like segments using unsupervised term discovery (UTD).
This information provides constraints that can be use to find speech frame representations that are more stable within a given phoneme.
The rationale is that while at the frame level the same speech sound can seem quite different between different speakers or even different realisations of the sound by the same speaker, patterns over a longer duration of time are easier to identify; this idea is illustrated in \textcite{jansen2013weak}.

The UTD systems used in this context are generally based on the segmental dynamic time warping (S-DTW) developed by \textcite{park2008unsupervised}.
S-DTW works by repeatedly performing DTW on two audio streams while constraining the maximum amount of warping allowed, each time changing the starting point of the DTW in both streams.
This yields a set of alignments, from which the stretches of lowest average dissimilarity in each alignment can be extracted.
Unfortunately, this approach is inherently $O(n^2)$ in time.
To remedy this, \textcite{jansen2011efficient} introduced an approximate version that uses binary approximations of the feature vectors to perform the calculations in $O(n \log n)$ time using sparse similarity matrices; this system also serves as the baseline for the second track of the Zero Resource Speech Challenge \parencite{versteegh2015zero}.

\textcite{jansen2011towards} describe a method for finding subword units, assuming that clusters corresponding to words, each cluster containing multiple examples of that word in the form of audio, are given.
For each word, an HMM is trained on all the corresponding examples, the number of states in the model being set to a number proportional to the average duration of the word.
The states from each HMM are then collected and clustered based on the similarity of their distributions, forming clusters that hopefully correspond to subword units.

\textcite{jansen2013weak} take somewhat of an inverse approach, starting by clustering the whole data on a frame level, with the assumption that each cluster will tend to correspond to some speaker- or context-dependent subword unit.
They then look at pairs of word-like segments known to be of the same type and calculate how often clusters tend to co-occurr.
The clusters are then partitioned so that clusters that co-occurr often are placed in the same partition.

\textcite{synnaeve2014phonetics} introduce a neural network known referred to as the ABnet, based on siamese networks \parencite{bromley1994signature}.
The network takes a pair of speech frames as input, and adjusts its parameters so that the outputs are collinear if the inputs are known to correspond to the same subword unit, and orthogonal otherwise, using a cosine-based loss function.
\textcite{thiolliere2015hybrid} made use of this approach in the Zero Resource Speech Challenge, also incorporating unsupervised term discovery so as to make the whole process unsupervised, yielding competitive results \parencite{versteegh2016zero}.
\textcite{zeghidour2016deep} experiment with supplying the ABnet with scattering spectrum features instead of filter bank features, showing that with the right features, a shallow architecture may outperform a deep architecture, especially when the amount of available data is low.

\textcite{kamper2015unsupervised} use an autoencoder-like structure, where a neural network is trained to ``reconstruct'' a frame given another frame known to be of the same type.
\textcite{renshaw2015comparison} used this architecture in the Zero Resource Speech Challenge, albeit with a deeper decoder.
